\documentclass[11pt]{book}
\usepackage[margin=1in]{geometry}
\usepackage{amsmath}
\usepackage{graphicx}
\usepackage{hyperref}
\usepackage[numbers]{natbib}
\usepackage{url}
\usepackage{microtype}
\usepackage{url}
\usepackage{microtype}

\setlength{\parindent}{0pt}
\setlength{\parskip}{0.6\baselineskip}

\title{AI-Driven E-Commerce (Master's Level)\\Book Structure \& 14-Week Plan (Placeholder)}
\author{Faculty of Computers and Information Sciences\\Mansoura University, Egypt}
\date{\today}

\begin{document}
\maketitle
\tableofcontents

\chapter*{How to Use This Draft}
This document is a \emph{structure-only} placeholder for an AI-driven, practical Master's-level e-commerce book.
Each chapter will be written later; for now, sections contain brief bullets stating scope, key skills, and deliverables.

\chapter*{Intended Audience and Prerequisites}
\begin{itemize}
  \item Audience: Master's students in Information Systems / Computer \& Information Sciences.
  \item Prereqs (recommended): databases, basic programming, web fundamentals, and introductory ML.
  \item Tooling (suggested): Python, notebooks, APIs, cloud services, and an ERP/EA modeling tool (as available).
\end{itemize}

\chapter*{Hands-on Track (Agreed)}
\begin{itemize}
  \item Approach: both tracks (lighter each).
  \item Track A (Enterprise/ERP): Odoo-focused integration labs (Sales/Inventory/Accounting flows).
  \item Track B (AI/ML): ML-heavy notebooks using e-commerce datasets (recsys, pricing, fraud, support).
\end{itemize}

\chapter*{Course Outcomes (Draft)}
\begin{itemize}
  \item Design end-to-end e-commerce systems with enterprise architecture (EA) and integration concerns.
  \item Apply ML/DL/AI to personalization, search, pricing, fraud, operations, and customer service.
  \item Integrate e-commerce with ERP/CRM/SCM (reference: Odoo) via APIs, events, and modern integration patterns.
  \item Evaluate systems for security, privacy, governance, ethics, and measurable business value.
\end{itemize}

\part{14-Week Teaching Plan (Mapping Weeks to Chapters)}
\chapter{Weekly Plan Overview}
\section*{Course emphasis (agreed)}
\begin{itemize}
  \item Overall emphasis: balanced (EA/ERP + ML/DL/AI).
  \item ERP reference stack: Odoo (concepts remain vendor-neutral where possible).
\end{itemize}
\begin{center}
\begin{tabular}{|p{0.12\linewidth}|p{0.80\linewidth}|}
  \hline
  Week & Topic (maps to the corresponding chapter) \\
  \hline
  1 & Foundations of modern e-commerce systems and AI-driven product thinking \\
  2 & Digital platforms, marketplaces, and business models (B2C/B2B/D2C, multi-sided platforms) \\
  3 & E-commerce data foundations: events, tracking, data quality, and experimentation \\
  4 & Enterprise architecture for e-commerce: capabilities, domains, and reference architectures \\
  5 & Enterprise integration: APIs, iPaaS, ESB, events/streaming, and integration patterns \\
  6 & ERP/CRM/SCM integration: order-to-cash, procure-to-pay, inventory, and master data \\
  7 & Recommenders I: classical + learning-to-rank for personalization and search \\
  8 & Recommenders II: deep learning, embeddings, sequence models, and retrieval \\
  9 & Pricing and promotion analytics: forecasting, elasticity, and optimization (ML + OR) \\
  10 & Fraud, risk, and trust \& safety: anomaly detection, graph ML, and AML patterns \\
  11 & Customer service automation: LLMs, RAG, chatbots, and agentic workflows \\
  12 & MLOps + DataOps for e-commerce: deployment, monitoring, drift, and governance \\
  13 & Security, privacy, compliance, and responsible AI in commerce \\
  14 & Capstone: enterprise-grade AI e-commerce solution architecture + evaluation \\
  \hline
\end{tabular}
\end{center}

\part{Book Structure (Chapters with Placeholders)}
\chapter{An Introduction to E-Commerce with a Focus on Machine Learning, Deep Learning, and Artificial Intelligence}

\section*{Abstract}
Electronic commerce (e-commerce) has evolved from static online catalogs into highly dynamic, data-driven ecosystems.
This transformation has been powered largely by advances in artificial intelligence (AI), particularly machine learning (ML) and deep learning (DL).
This chapter introduces the foundations of e-commerce and explains how AI techniques are embedded across the e-commerce value chain---from customer acquisition and product discovery to pricing, logistics, and fraud prevention.
The discussion is aimed at master's students in computer and information sciences, and therefore emphasizes conceptual clarity, system-level thinking, and the mapping between theoretical models and real-world e-commerce applications.

\textbf{Keywords:} e-commerce, artificial intelligence, machine learning, deep learning, recommendation systems, dynamic pricing, fraud detection, personalization, logistics optimization

\section{Introduction to E-Commerce}
E-commerce refers to the buying and selling of goods and services via electronic networks, primarily the internet \citep{laudon2024ecommerce}.
It encompasses a broad set of transactional models: business-to-consumer (B2C) online retail, business-to-business (B2B) procurement platforms, consumer-to-consumer (C2C) marketplaces, and consumer-to-business (C2B) models such as influencer platforms and freelance marketplaces \citep{laudon2024ecommerce}.

From a computing perspective, an e-commerce platform is not just a web front-end.
It is a socio-technical system composed of:
\begin{itemize}
  \item User interfaces (web, mobile, conversational)
  \item Application logic (catalog, cart, checkout, account management)
  \item Payment and risk engines
  \item Logistics and fulfillment systems
  \item Data infrastructure and analytics pipelines
  \item AI/ML components that drive personalization, prediction, and automation
\end{itemize}

A key characteristic of e-commerce is its data richness.
Every user interaction---page views, searches, clicks, scrolls, wishlists, add-to-carts, purchases, returns, and even time-to-decision---can be captured as fine-grained behavioral logs.
Combined with product metadata, prices, promotions, and external contextual data, this creates an ideal environment for AI and ML \citep{laudon2024ecommerce,kleppmann2017ddia}.

\subsection{E-Commerce Business Models and Value Chain}
Common e-commerce models include:
\begin{itemize}
  \item \textbf{B2C retail:} Online stores selling directly to consumers.
  \item \textbf{Marketplaces:} Platforms mediating between many sellers and buyers.
  \item \textbf{B2B platforms:} Portals for wholesale ordering, procurement, and supply chain integration.
  \item \textbf{Digital services:} Software subscriptions, digital content, and online education.
\end{itemize}

Across these models, the \textbf{e-commerce value chain} typically includes:
\begin{enumerate}
  \item Customer acquisition and traffic generation (marketing, ads, SEO)
  \item Product discovery and decision support (search, recommendations, reviews)
  \item Conversion and transaction processing (checkout, payments, risk checks)
  \item Order fulfillment and logistics (inventory, warehousing, routing)
  \item Post-purchase engagement (support, returns, loyalty, re-activation)
\end{enumerate}

\noindent AI, ML, and DL are now present at each stage of this chain \citep{laudon2024ecommerce}.

\section{Foundations of AI, Machine Learning, and Deep Learning}
\subsection{Definitions and Relationships}
\begin{itemize}
  \item \textbf{Artificial Intelligence (AI):} Systems that exhibit behaviors considered ``intelligent'' (perception, reasoning, learning, decision-making).
  \item \textbf{Machine Learning (ML):} Algorithms that learn patterns from data and improve with experience.
  \item \textbf{Deep Learning (DL):} ML methods based on multi-layer neural networks that learn representations from large-scale data.
\end{itemize}
Thus, deep learning $\subset$ machine learning $\subset$ artificial intelligence \citep{goodfellow2016dl,murphy2022ml,hastie2009esl}.

\subsection{Types of Learning and E-Commerce Examples}
\begin{enumerate}
  \item \textbf{Supervised learning:} learn $f:X\rightarrow Y$ from labeled examples $(x_i,y_i)$.
  Examples: conversion prediction, fraud classification, demand forecasting.
  \item \textbf{Unsupervised learning:} discover structure in unlabeled data.
  Examples: customer segmentation, anomaly detection, and learning embeddings from interactions.
  \item \textbf{Reinforcement learning:} learn a policy to maximize long-term reward.
  Examples: recommendation policies and dynamic pricing under constraints.
  \item \textbf{Representation learning:} learn features $\phi(x)$ automatically (often via DL).
  Examples: user/item embeddings; Transformer-based session representations.
\end{enumerate}

\section{Machine Learning Across the E-Commerce Lifecycle}
\subsection{Personalized Recommendations}
Recommendation systems use past behavior and product attributes to suggest items a user is likely to buy \citep{aggarwal2016recsys,koren2009mf}.
Key evaluation ideas include ranking quality (precision@$k$, recall@$k$, NDCG) and business impact (incremental revenue) \citep{jannach2017recsys_eval}.

\subsection{Search and Ranking}
Learning-to-rank uses user feedback (including clicks) to optimize product ranking in response to queries \citep{joachims2002ltr}.
Modern systems increasingly incorporate semantic retrieval and NLP components \citep{jurafsky2023slp}.

\subsection{Segmentation, churn, and CLV}
Segmentation (e.g., via clustering) and predictive modeling (e.g., churn/CLV) connect behavioral data to marketing and retention actions \citep{hastie2009esl}.

\subsection{Dynamic Pricing and Promotion Optimization}
Pricing combines prediction (demand response) with optimization under constraints; evaluation typically requires controlled online experimentation \citep{laudon2024ecommerce,hastie2009esl}.

\subsection{Demand Forecasting and Inventory Management}
Forecasting methods range from statistical time-series models to ML sequence models; forecasts drive replenishment and inventory decisions \citep{ngai2011ec,hastie2009esl}.

\subsection{Fraud Detection and Transaction Security}
Fraud detection is an imbalanced classification problem with non-stationarity (attackers adapt), and must balance false positives against fraud loss \citep{dalpozzolo2015fraud,hastie2009esl}.

\subsection{Customer Service and Conversational Commerce}
Customer support automation mixes information retrieval, intent/entity extraction, and increasingly LLM-based interaction \citep{jurafsky2023slp}.

\section{Deep Learning in E-Commerce}
DL is especially useful for unstructured and high-dimensional data (text, images, sequences) \citep{goodfellow2016dl}.

\subsection{Neural Recommendation Systems}
Deep recommender architectures model non-linear user--item interactions and session dynamics \citep{covington2016youtube,hidasi2016gru4rec}.

\subsection{NLP for Product Text and Reviews}
NLP supports query understanding, semantic search, review analysis, and conversational interfaces \citep{jurafsky2023slp}.

\subsection{Computer Vision and Visual Search}
Vision models support category classification, attribute extraction, and similarity search via embeddings \citep{goodfellow2016dl}.

\section{Data, Architecture, and Deployment Considerations}
E-commerce ML requires robust pipelines (data quality, governance), reliable deployment (latency, availability), and continuous evaluation (monitoring and experiments) \citep{kleppmann2017ddia}.

\section{Challenges and Responsible AI}
Key challenges include bias, privacy/security, explainability for high-impact decisions, and the organizational processes needed for safe deployment \citep{laudon2024ecommerce,kleppmann2017ddia}.

\section{Multiple-choice questions (MCQs)}
\begin{enumerate}
  \item Which relationship is correct?
  \begin{enumerate}
    \item AI $\subset$ ML $\subset$ DL
    \item DL $\subset$ ML $\subset$ AI
    \item ML $\subset$ DL $\subset$ AI
    \item AI, ML, and DL are disjoint fields
  \end{enumerate}

  \item In e-commerce, which is \textbf{most} naturally framed as a \emph{ranking} problem?
  \begin{enumerate}
    \item Choosing the best warehouse location for a new facility
    \item Ordering products in a search results page for a query
    \item Estimating the total demand for next quarter
    \item Balancing accounting entries for an invoice
  \end{enumerate}

  \item Which metric is \textbf{most} aligned with offline evaluation of a top-$k$ recommender?
  \begin{enumerate}
    \item NDCG@$k$
    \item Mean squared error (MSE) on prices
    \item Page load time (ms)
    \item Uptime percentage
  \end{enumerate}

  \item Fraud detection is often difficult because:
  \begin{enumerate}
    \item Fraud labels are always perfectly accurate
    \item The problem is typically extremely imbalanced and attackers adapt over time
    \item Fraud has no economic cost, only technical cost
    \item The best approach is always rule-based
  \end{enumerate}

  \item Which statement best describes why A/B testing is important in e-commerce ML?
  \begin{enumerate}
    \item It guarantees the model is unbiased
    \item It measures causal impact on business KPIs under real user behavior
    \item It replaces the need for offline evaluation entirely
    \item It eliminates the need for monitoring after deployment
  \end{enumerate}
\end{enumerate}

\subsection*{Answer key}
\begin{enumerate}
  \item (b)
  \item (b)
  \item (a)
  \item (b)
  \item (b)
\end{enumerate}
\chapter{An Introduction to E-Commerce with a Focus on Machine Learning, Deep Learning, and Artificial Intelligence}


\begin{abstract}
Electronic commerce (e-commerce) has evolved from static online catalogs into highly dynamic, data-driven ecosystems.
This transformation has been powered largely by advances in artificial intelligence (AI), particularly machine learning (ML) and deep learning (DL).
This chapter introduces the foundations of e-commerce and explains how AI techniques are embedded across the e-commerce value chain---from customer acquisition and product discovery to pricing, logistics, and fraud prevention.
The discussion is aimed at master's students in computer and information sciences, and therefore emphasizes conceptual clarity, system-level thinking, and the mapping between theoretical models and real-world e-commerce applications.
\end{abstract}

\textbf{Keywords:} E-commerce, Artificial Intelligence, Machine Learning, Deep Learning, Recommendation Systems, Dynamic Pricing, Fraud Detection, Personalization, Logistics Optimization

\section{Introduction to E-Commerce}
E-commerce refers to the buying and selling of goods and services via electronic networks, primarily the internet.
It encompasses a broad set of transactional models: business-to-consumer (B2C) online retail, business-to-business (B2B) procurement platforms, consumer-to-consumer (C2C) marketplaces, and consumer-to-business (C2B) models such as influencer platforms and freelance marketplaces.

\citep{usal_ecommerce_ml_2024}

\citep{usal_ecommerce_ml_2024}


From a computing perspective, an e-commerce platform is not just a web front-end.
It is a socio-technical system composed of:
\begin{itemize}
  \item User interfaces (web, mobile, conversational)
  \item Application logic (catalog, cart, checkout, account management)
  \item Payment and risk engines
  \item Logistics and fulfillment systems
  \item Data infrastructure and analytics pipelines
  \item AI/ML components that drive personalization, prediction, and automation
\end{itemize}

A key characteristic of e-commerce is its data richness.
Every user interaction---page views, searches, clicks, scrolls, wishlists, add-to-carts, purchases, returns, and even time-to-decision---can be captured as fine-grained behavioral logs.Combined with product metadata, prices, promotions, and external contextual data, this creates an ideal environment for AI and ML.
\citep{nix_ml_ecommerce}



\subsection{E-Commerce Business Models and Value Chain}
Common e-commerce models include:
\begin{itemize}
  \item \textbf{B2C Retail:} Online stores selling directly to consumers (e.g., Amazon, local grocery delivery).
  \item \textbf{Marketplaces:} Platforms mediating between many sellers and buyers (e.g., eBay, regional platforms).
  \item \textbf{B2B Platforms:} Portals for wholesale ordering, procurement, and supply chain integration.
  \item \textbf{Digital Services:} Software subscriptions, digital content, and online education.
\end{itemize}

Across these models, the \textbf{e-commerce value chain} typically includes:
\begin{enumerate}
  \item Customer acquisition and traffic generation (marketing, ads, SEO).
  \item Product discovery and decision support (search, recommendations, reviews).
  \item Conversion and transaction processing (checkout, payments, risk checks).
  \item Order fulfillment and logistics (inventory, warehousing, routing).
  \item Post-purchase engagement (support, returns, loyalty, re-activation\end{enumerate}
\noindent AI, ML, and DL are now present at each stage of this chain.\citep{univio_ai_ecommerce}

}

\section{Foundations of AI, Machine Learning, and Deep Learning}
\subsection{Definitions and Relationships}
\begin{itemize}
  \item \textbf{Artificial Intelligence (AI):} The broader field focused on building systems that exhibit behaviors considered ``intelligent,'' such as perception, reasoning, learning, and decision-making.
  \item \textbf{Machine Learning (ML):} A subfield of AI that develops algorithms which learn patterns from data and improve with experience, rather than relying on explicit, hand-crafted rules.
  \item \textbf{Deep Learning (DL):} A subfield of ML based on artificial neural networks with many layers, capable of automatically learning complex representations from large volumes of data, particularly unstructured data such as images, text, and click sequences.
\end{itemize}
ThThus, deep learning $\subset$ machine learning $\subset$ artificial intelligence.
\citep{jcer_ai_ml_dl_ebook,raschka_intro_dl_ch01,routledge_ai_freebook}

\subsection{Types of Learning and E-Commerce Examples}
\begin{enumerate}
  \item \textbf{Supervised learning:} Learn a mapping $f:X\rightarrow Y$ from labeled examples $(x_i,y_i)$.
  Examples: conversion prediction, fraud classification, demand forecasting.
  \item \textbf{Unsupervised learning:} Discover structure in unlabeled data.
  Examples: customer clustering, anomaly detection, learning product embeddings from co-view/co-purchase graphs.
  \item \textbf{Reinforcement learning (RL):} Learn a policy that maps states to actions to maximize long-term reward.
  Examples: adaptive recommendation strategies; dynamic pricing agents balancing revenue, churn, and fairness constraints.
  \item \textbf{Representation learning (often via DL):} Learn feature representations $\phi(x)$ automatically.
  Examples: user/item embeddings; Transformer-based session models.
\end{enumerate}

\noindent AI, ML, and DL are now present at each stage of this chain.\citep{univio_ai_ecommerce}

\section{Machine Learning Across the E-Commerce LifecycML algorithms are now embedded in nearly all functional domains of e-commerce, including recommendation, fraud detection, demand forecasting, pricing, and customer segmentation.
\citep{nix_ml_ecommerce}

n.

\subsection{Personalized Recommendations}
Recommendation systems use past behavior and product attributes to suggest items a user is likely to buy.
Typical approaches include:
\begin{itemize}
  \item \textbf{Collaborative filtering:} Uses an interaction matrix $R=[r_{ui}]$.
  Memory-based $k$-nearest neighbors over users or items; model-based matrix factorization:
  \[
    \hat{r}_{ui}=\mathbf{p}_u^\top \mathbf{q}_i,
  \]
  where $\mathbf{p}_u$ and $\mathbf{q}_i$ are latent factors for user and item.
  \item \textbf{Content-based filtering:} Uses item features (category, brand, text description, image features) and user profiles to recommend similar items.
  \item \textbf{Hybrid models:} Combine collaborative and content-based signals, often via gradient boosted trees or neural ranking models.
\end{itemCommon evaluation metrics include precision@$k$, recall@$k$, NDCG, click-through rate (CTR), and incremental revenue.
\citep{gfg_ml_usecases_ecommerce}

ue.

\subsection{Search and Ranking}
E-commerce search engines retrieve and rank products based on user queries.
ML enters at multiple layers: query understanding (spell correction, synonyms, intent classification), learning-to-rank, and personalized re-ranDeep learning methods increasingly support semantic search by encoding queries and products into dense embeddings.
\citep{ieee_ecommerce_ai_11009009}

ngs.

\subsection{Customer Segmentation and Lifetime Value}
Customer segmentation partitions users into groups with similar behavioral patterns, enabling tailored marketing and personalization strategies.
Clustering methods include $k$-means, Gaussian mixture models, and density-based clustering using features such as recency, frequency, and monetary value Supervised models estimate churn probability and customer lifetime value (CLV), often approximated by the discounted sum of expected future purchases.
\citep{usal_ecommerce_ml_2024,nix_ml_ecommerce}

ases.

\subsection{Dynamic Pricing and Promotion Optimization}
ML models can predict demand as a function of price, promotions, competitor prices, seasonality, and user seTypical components include demand modeling (regression/boosted trees), constrained optimization for price setting, and online experimentation (A/B testing and bandits).
\citep{univio_ai_ecommerce,nix_ml_ecommerce}

dits).

\subsection{Demand Forecasting and Inventory Management}
Forecasting models integrate historical sales, seasonality, marketing campaigns, product lifecycle stages, and external Approaches range from ARIMA/exponential smoothing to gradient boosting and sequence models; forecasts feed inventory planning, warehouse allocation, and supplier ordering.
\citep{univio_ai_ecommerce}

dering.

\subsection{Fraud Detection and Transaction Security}
Fraud and account takeover are major risks in e-commerce.
Rule-based systems struggle to adapt to new attack patterns; ML models learn from high-dimensional features such as device fingerprints, geolocation, velocity indicators, and userSupervised learning is complemented by unsupervised and semi-supervised anomaly detection, and by graph-based methods in relational settings.
\citep{nexocode_fraud_ml,unl_fraud_slr}

ettings.

\subsection{Customer Service, Chatbots, and Virtual Assistants}
AI-powered chatbots and virtual assistants support customers 24/7, answering queries about orders, returns, product details, and troublTypical components include intent classification, named entity recognition (NER), dialogue management, and response generation via templates, retrieval, or generative language models.
\citep{univio_ai_ecommerce,gfg_ml_usecases_ecommerce}

e models.

\section{Deep Learning in Deep learning plays a central role in handling large-scale, high-dimensional, and unstructured data typical of modern e-commerce systems.
\citep{usal_ecommerce_ml_2024}

e systems.

\subsection{Neural Recommendation Systems}
Deep recommendation models can replace the dot product $\mathbf{p}_u^\top \mathbf{q}_i$ with a neural network $f_\theta(\mathbf{p}_u,\mathbf{q}_i)$ to learn non-linear iSequence-based recommenders (RNNs/Transformers) model user sessions; context-aware models incorporate device, time, location, and campaign signals.
\citep{ieee_ecommerce_ai_11009009}

gn signals.

\subsection{Natural Language Processing (NLP) for DL-based NLP supports review analysis (sentiment and aspects), query understanding and semantic search, and conversational AI for multi-turn customer support.
\citep{ieee_ecommerce_ai_11009009}

mer support.

\subsection{Computer Vision and Computer vision enables image classification, object detection, visual search via embeddings, and content moderation for catalog integrity.
\citep{gfg_ml_usecases_ecommerce}

og integrity.

\subsection{Sequential and Graph Modeling ofSequence models capture temporal dynamics and short-term intent; graph neural networks (GNNs) capture interaction graphs connecting users, items, categories, and attributes for recommendation and fraud detection.
\citep{unl_fraud_slr}

aud detection.

\subsection{Deep Learning for Deep models help address class imbalance and temporal/adversarial patterns by learning non-linear boundaries and by incorporating sequences and graphs for real-time detection.
\citep{iapress_fraud_dl}

time detection.

\section{AI for Operations, Logistics, and Marketing Optimization}
\subsection{Logistics Applications include route optimization (ML + combinatorial optimization), warehouse automation (robots and vision), and inventory placement across warehouses.
\citep{ttms_ai_ecommerce}

ross warehouses.

\subsection{Marketing and CaML supports response prediction, channel/budget optimization, and personalization; generative AI can assist with product descriptions, localization, and SEO-oriented text generation.
\citep{univio_ai_ecommerce}

 text generation.

\section{Data, Architecture, and Deployment Considerations}
\subsection{Data Sources and Feature Engineering}
Typical data sources include behavioral logs, transactional records, the product catalog, customer data (where permitted), and Feature engineering transforms raw logs into model-ready features; DL reduces reliance on manual features but still benefits from strong input design.
\citep{nix_ml_ecommerce}

rong input design.

\subsection{Offline Training vs.\ Online Serving}
Most pipelines have an offline phase (extraction, cleaning, training, tuning, offline evaluation) and an online phase (real-time features, serving, and Streaming architectures and low-latency feature stores support real-time personalization and fraud detection.
\citep{nix_ml_ecommerce}

nd fraud detection.

\subsection{Evaluation, Experimentation, and Monitoring}
Offline metrics (AUC, precision@$k$, RMSE) are necessary but insufficient; online KPIs include conversion rate, average order value, revenue per session, retention,A/B testing is the standard for causal assessment; deployed models must be monitored for drift, bias, and performance degradation.
\citep{ieee_ecommerce_ai_11009009}

ormance degradation.

\section{Challenges, Risks, and Ethical Considerations}
\begin{enumerate}
  \item \textbf{Data quality and bias:} Historical data can encode bias; models can amplify unfair outcomes.
  \item \textbf{Privacy and security:} Collection and processing of user data raise privacy obligations; breaches and model attacks are risks.
  \item \textbf{Explainability and transparency:} High-impact decisions (fraud blocking, credit) require interpretable policies and explanations.
  \item \textbf{Scalability and cost:} Large models and low-latency serving require careful systems trade-offs.
  \item \textbf{Organizational readiness:} Successful deployment requires culture, governance, and MLOps practices, not only algorithms.
\end{enumerate}

\section{Future Directions inProminent directions include generative AI for commerce, causal inference and uplift modeling, reinforcement learning at scale, multimodal and unified models, deeper use of graphs, and sustainability-aware optimization.
\citep{univio_ai_ecommerce,ieee_ecommerce_ai_11009009}

y-aware optimization.

\section{Summary and Pedagogical Notes}
This chapter introduced e-commerce as a data-rich, AI-intensive domain; clarified AI/ML/DL relationships; surveyed ML/DL applications; and highlighted deployment, evaluation, and ethical issues.
Suggested follow-up activities include implementing a basic recommender, designing a fraud detection pipeline for imbalanced data, simulating A/B tests for ranking/pricing strategies, and analyzing an AI feature from technical and ethical perspectives.

\chapter{Platforms, Marketplaces, and Business Models (Week 2)}

\section{Learning objectives}
\begin{itemize}
  \item Distinguish e-commerce \emph{channels} (storefronts) from \emph{platforms} (ecosystems) and \emph{marketplaces} (multi-seller).
  \item Analyze platform economics: network effects, multi-homing, and pricing of participation.
  \item Translate business model choices into architectural requirements (identity, trust, data, integration, ERP).
\end{itemize}

\section{From storefront to platform}
\begin{itemize}
  \item \textbf{Storefront:} a single merchant selling to buyers (B2C/B2B) via a web/app channel.
  \item \textbf{Platform:} a system that enables interactions among multiple participant groups (e.g., buyers, sellers, advertisers, logistics providers).
  \item \textbf{Marketplace:} a platform where multiple sellers list items and transactions are mediated by platform policies and infrastructure.
\end{itemize}

\section{Value creation, value capture, and growth}
\subsection{Value creation and matching}
Platforms create value by reducing search and transaction costs, increasing variety, and improving trust; AI frequently strengthens these effects by improving matching and decision support.

\subsection{Value capture models}
\begin{itemize}
  \item Take rate (commission on completed transactions)
  \item Subscription (seller or buyer membership tiers)
  \item Listing fees and value-added services (e.g., promotions, analytics)
  \item Advertising (sponsored listings, retail media)
  \item Fulfillment fees (warehousing, last-mile delivery, returns handling)
\end{itemize}

\subsection{Network effects}
\begin{itemize}
  \item \textbf{Direct network effects:} more users attract more users (e.g., social commerce communities).
  \item \textbf{Indirect network effects:} more buyers attract more sellers and vice versa (classic marketplace dynamic).
  \item \textbf{Cold start:} bootstrapping supply and demand when network effects are weak.
  \item \textbf{Multi-homing:} sellers/buyers participate in multiple platforms; affects differentiation strategy.
\end{itemize}

\section{Governance and trust as system requirements}
Marketplace governance is not ``policy only'': it becomes a set of product and system requirements.
\begin{itemize}
  \item Seller onboarding, verification, and compliance workflows (KYC/KYB where relevant).
  \item Catalog integrity: listing policies, moderation, and counterfeit detection.
  \item Reviews/ratings, dispute resolution, refunds/returns policy enforcement.
  \item Trust \& safety operations: rule+ML detection, case management, and human-in-the-loop escalation.
\end{itemize}

\section{Business model patterns (and architectural implications)}
\subsection{B2C retail}
\begin{itemize}
  \item Key systems: PIM/catalog, promotions, payments, fulfillment, customer support.
  \item ERP/CRM integration: order-to-cash, inventory, accounting, returns.
\end{itemize}

\subsection{B2B commerce}
\begin{itemize}
  \item Quotes, negotiated pricing, approval workflows, invoicing, credit limits.
  \item Integration-heavy: procurement, supplier management, and EDI/API flows.
\end{itemize}

\subsection{Marketplace}
\begin{itemize}
  \item Split catalog ownership (platform vs sellers), seller tools, payout/settlement.
  \item Additional ``control plane'': seller policies, listing moderation, abuse prevention.
\end{itemize}

\subsection{Digital services (subscriptions)}
\begin{itemize}
  \item Recurring billing, entitlements, and churn/retention analytics.
  \item Usage-based pricing requires event design and metering.
\end{itemize}

\section{Where AI fits (system-level view)}
\begin{itemize}
  \item \textbf{Matching layer:} search, recommendations, ranking.
  \item \textbf{Trust layer:} fraud/abuse detection, content moderation.
  \item \textbf{Operations layer:} forecasting, replenishment, routing.
  \item \textbf{Experience layer:} personalization and conversational commerce.
\end{itemize}

\section{Hands-on (placeholder)}
\begin{itemize}
  \item Create a one-page \emph{platform blueprint}: participant groups, value exchange, pricing model, and required capabilities.
  \item Map capabilities to an EA view and identify which capabilities are ERP-owned vs commerce-owned.
\end{itemize}

\section{Case studies (placeholder)}
\begin{itemize}
  \item Modern marketplace patterns (generic; no vendor lock-in).
\end{itemize}

\section{Multiple-choice questions (MCQs)}
\begin{enumerate}
  \item Which statement best distinguishes a \emph{storefront} from a \emph{marketplace}?
  \begin{enumerate}
    \item A storefront is always B2B; a marketplace is always B2C.
    \item A storefront has a single merchant of record; a marketplace mediates transactions among multiple sellers and buyers.
    \item A storefront cannot use AI; a marketplace must use AI.
    \item A storefront requires ERP integration; a marketplace does not.
  \end{enumerate}

  \item \emph{Indirect network effects} in marketplaces most directly mean:
  \begin{enumerate}
    \item The platform has a single user group and growth is linear.
    \item The value to buyers increases as more sellers join, and the value to sellers increases as more buyers join.
    \item Users prefer multiple platforms (multi-homing) regardless of differentiation.
    \item The platform must subsidize logistics to succeed.
  \end{enumerate}

  \item In B2B commerce, which requirement is \textbf{most typical} compared to B2C?
  \begin{enumerate}
    \item Session-based recommendations
    \item Negotiated pricing and approval workflows
    \item Visual search
    \item Social reviews and influencer marketing
  \end{enumerate}

  \item \emph{Multi-homing} refers to:
  \begin{enumerate}
    \item Hosting the platform in multiple cloud regions.
    \item Sellers or buyers participating in multiple platforms simultaneously.
    \item Using multiple payment gateways for redundancy.
    \item Maintaining multiple warehouses for the same SKU.
  \end{enumerate}

  \item Which is the \textbf{best} example of a marketplace \emph{governance} capability?
  \begin{enumerate}
    \item Re-ranking items using embeddings
    \item Seller onboarding verification and dispute resolution policies
    \item Demand forecasting for replenishment
    \item Offline model training with hyperparameter tuning
  \end{enumerate}
\end{enumerate}

\subsection*{Answer key}
\begin{enumerate}
  \item (b)
  \item (b)
  \item (b)
  \item (b)
  \item (b)
\end{enumerate}
\chapter{Platforms, Marketplaces, and Business Models (Week 2)}

\section{Learning objectives}
\begin{itemize}
  \item Distinguish e-commerce \emph{channels} (storefronts) from \emph{platforms} (ecosystems) and \emph{marketplaces} (multi-seller).
  \item Analyze platform economics: network effects, multi-homing, and pricing of participation.
  \item Translate business model choices into architectural requirements (identity, trust, data, integration, ERP).
\end{itemize}


\section{Multiple-choice questions (MCQs)}
\begin{enumerate}
  \item Which statement best distinguishes a \emph{storefront} from a \emph{marketplace}?
  \begin{enumerate}
    \item A storefront is always B2B; a marketplace is always B2C.
    \item A storefront has a single merchant of record; a marketplace mediates transactions among multiple sellers and buyers.
    \item A storefront cannot use AI; a marketplace must use AI.
    \item A storefront requires ERP integration; a marketplace does not.
  \end{enumerate}

  \item \emph{Indirect network effects} in marketplaces most directly mean:
  \begin{enumerate}
    \item The platform has a single user group and growth is linear.
    \item The value to buyers increases as more sellers join, and the value to sellers increases as more buyers join.
    \item Users prefer multiple platforms (multi-homing) regardless of differentiation.
    \item The platform must subsidize logistics to succeed.
  \end{enumerate}

  \item In B2B commerce, which requirement is \textbf{most typical} compared to B2C?
  \begin{enumerate}
    \item Session-based recommendations
    \item Negotiated pricing and approval workflows
    \item Visual search
    \item Social reviews and influencer marketing
  \end{enumerate}

  \item \emph{Multi-homing} refers to:
  \begin{enumerate}
    \item Hosting the platform in multiple cloud regions.
    \item Sellers or buyers participating in multiple platforms simultaneously.
    \item Using multiple payment gateways for redundancy.
    \item Maintaining multiple warehouses for the same SKU.
  \end{enumerate}

  \item Which is the \textbf{best} example of a marketplace \emph{governance} capability?
  \begin{enumerate}
    \item Re-ranking items using embeddings
    \item Seller onboarding verification and dispute resolution policies
    \item Demand forecasting for replenishment
    \item Offline model training with hyperparameter tuning
  \end{enumerate}
\end{enumerate}

\subsection*{Answer key}
\begin{enumerate}
  \item (b)
  \item (b)
  \item (b)
  \item (b)
  \item (b)
\end{enumerate}

\section{Scope}
\begin{itemize}

  \item Multi-sided platforms, network effects, platform governance, and incentives.
  \item B2B e-commerce, procurement portals, and industry-specific marketplaces.\end{itemize}
\section{Core concepts for Master's students}
\subsection{From storefront to platform}
\begin{itemize}
  \item \textbf{Storefront:} a single merchant selling to buyers (B2C/B2B) via a web/app channel.
  \item \textbf{Platform:} a system that enables interactions among multiple participant groups (e.g., buyers, sellers, advertisers, logistics providers).
  \item \textbf{Marketplace:} a platform where multiple sellers list items and transactions are mediated by platform policies and infrastructure.
\end{itemize}

\subsection{Value creation and capture}
\begin{itemize}
  \item \textbf{Value creation:} reduce search/transaction costs, increase variety, improve trust, and improve matching (often AI-driven).
  \item \textbf{Value capture:} take rate, subscription, listing fees, ads, fulfillment fees, and data/analytics products.
\end{itemize}

\subsection{Network effects and growth loops}
\begin{itemize}
  \item \textbf{Direct network effects:} more users attract more users (e.g., social commerce communities).
  \item \textbf{Indirect network effects:} more buyers attract more sellers and vice versa (classic marketplace dynamic).
  \item \textbf{Cold start:} early-stage bootstrapping strategies (seed supply vs seed demand).
  \item \textbf{Multi-homing:} sellers/buyers participating in multiple platforms; implications for loyalty and differentiation.
\end{itemize}

\subsection{Governance and trust}
\begin{itemize}
  \item Identity and onboarding (KYC/KYB where relevant), seller verification, and catalog integrity.
  \item Reviews/ratings, dispute management, returns policy, and service-level enforcement.
  \item Trust \& safety as an \emph{operating system} for the marketplace: policies + detection + human-in-the-loop.
\end{itemize}

\section{Business model patterns (with architectural implications)}
\subsection{B2C retail}
\begin{itemize}
  \item Key systems: product information management (PIM), promotions, payments, fulfillment, customer service.
  \item ERP/CRM integration: order-to-cash, inventory, accounting, returns.
\end{itemize}

\subsection{B2B commerce}
\begin{itemize}
  \item Quotation, contracts, negotiated pricing, credit limits, invoicing, and approval workflows.
  \item Integration-heavy by nature: procurement, supplier management, and EDI/API flows.
\end{itemize}

\subsection{Marketplace}
\begin{itemize}
  \item Split catalog ownership (platform vs sellers), seller tools, and payout/settlement.
  \item Additional control plane: seller policies, listing moderation, and fraud/abuse prevention.
\end{itemize}

\subsection{Digital services (subscriptions)}
\begin{itemize}
  \item Recurring billing, entitlement management, and churn/retention analytics.
  \item Usage-based pricing requires event design and metering.
\end{itemize}

\section{AI is not ``a feature'': where it fits in the platform}
\begin{itemize}
  \item \textbf{Matching layer:} search, recommendations, and ranking across buyers and sellers.
  \item \textbf{Trust layer:} fraud detection, abuse detection, and content moderation.
  \item \textbf{Operations layer:} forecasting, replenishment, routing, and workforce planning.
  \item \textbf{Experience layer:} conversational commerce, assistants, and personalization.
\end{itemize}

\section{Hands-on (placeholder)}
\begin{itemize}
  \item Build a one-page \emph{platform blueprint}: participant groups, value exchange, pricing model, and required capabilities.
  \item Map the blueprint to a capability map and identify which capabilities are ERP-owned vs commerce-owned.
\end{itemize}

\section{Core concepts for Master's students}
\subsection{From storefront to platform}
\begin{itemize}
  \item \textbf{Storefront:} a single merchant selling to buyers (B2C/B2B) via a web/app channel.
  \item \textbf{Platform:} a system that enables interactions among multiple participant groups (e.g., buyers, sellers, advertisers, logistics providers).
  \item \textbf{Marketplace:} a platform where multiple sellers list items and transactions are mediated by platform policies and infrastructure.
\end{itemize}

\subsection{Value creation and capture}
\begin{itemize}
  \item \textbf{Value creation:} reduce search/transaction costs, increase variety, improve trust, and improve matching (often AI-driven).
  \item \textbf{Value capture:} take rate, subscription, listing fees, ads, fulfillment fees, and data/analytics products.
\end{itemize}

\subsection{Network effects and growth loops}
\begin{itemize}
  \item \textbf{Direct network effects:} more users attract more users (e.g., social commerce communities).
  \item \textbf{Indirect network effects:} more buyers attract more sellers and vice versa (classic marketplace dynamic).
  \item \textbf{Cold start:} early-stage bootstrapping strategies (seed supply vs seed demand).
  \item \textbf{Multi-homing:} sellers/buyers participating in multiple platforms; implications for loyalty and differentiation.
\end{itemize}

\subsection{Governance and trust}
\begin{itemize}
  \item Identity and onboarding (KYC/KYB where relevant), seller verification, and catalog integrity.
  \item Reviews/ratings, dispute management, returns policy, and service-level enforcement.
  \item Trust \& safety as an \emph{operating system} for the marketplace: policies + detection + human-in-the-loop.
\end{itemize}

\section{Business model patterns (with architectural implications)}
\subsection{B2C retail}
\begin{itemize}
  \item Key systems: product information management (PIM), promotions, payments, fulfillment, customer service.
  \item ERP/CRM integration: order-to-cash, inventory, accounting, returns.
\end{itemize}

\subsection{B2B commerce}
\begin{itemize}
  \item Quotation, contracts, negotiated pricing, credit limits, invoicing, and approval workflows.
  \item Integration-heavy by nature: procurement, supplier management, and EDI/API flows.
\end{itemize}

\subsection{Marketplace}
\begin{itemize}
  \item Split catalog ownership (platform vs sellers), seller tools, and payout/settlement.
  \item Additional control plane: seller policies, listing moderation, and fraud/abuse prevention.
\end{itemize}

\subsection{Digital services (subscriptions)}
\begin{itemize}
  \item Recurring billing, entitlement management, and churn/retention analytics.
  \item Usage-based pricing requires event design and metering.
\end{itemize}

\section{AI is not ``a feature'': where it fits in the platform}
\begin{itemize}
  \item \textbf{Matching layer:} search, recommendations, and ranking across buyers and sellers.
  \item \textbf{Trust layer:} fraud detection, abuse detection, and content moderation.
  \item \textbf{Operations layer:} forecasting, replenishment, routing, and workforce planning.
  \item \textbf{Experience layer:} conversational commerce, assistants, and personalization.
\end{itemize}

\section{Hands-on (placeholder)}
\begin{itemize}
  \item Build a one-page \emph{platform blueprint}: participant groups, value exchange, pricing model, and required capabilities.
  \item Map the blueprint to a capability map and identify which capabilities are ERP-owned vs commerce-owned.
\end{itemize}
\section{Case studies (placeholder)}
\begin{itemize}
}
  \item Modern marketplace patterns (generic, no vendor lock-in)\end{itemize}

\section{Multiple-choice questions (MCQs)}
\begin{enumerate}
  \item Which statement best distinguishes a \emph{storefront} from a \emph{marketplace}?
  \begin{enumerate}
    \item A storefront is always B2B; a marketplace is always B2C.
    \item A storefront has a single merchant of record; a marketplace mediates transactions among multiple sellers and buyers.
    \item A storefront cannot use AI; a marketplace must use AI.
    \item A storefront requires ERP integration; a marketplace does not.
  \end{enumerate}

  \item \emph{Indirect network effects} in marketplaces most directly mean:
  \begin{enumerate}
    \item The platform has a single user group and growth is linear.
    \item The value to buyers increases as more sellers join, and the value to sellers increases as more buyers join.
    \item Users prefer multiple platforms (multi-homing) regardless of differentiation.
    \item The platform must subsidize logistics to succeed.
  \end{enumerate}

  \item In B2B commerce, which requirement is \textbf{most typical} compared to B2C?
  \begin{enumerate}
    \item Session-based recommendations
    \item Negotiated pricing and approval workflows
    \item Visual search
    \item Social reviews and influencer marketing
  \end{enumerate}

  \item \emph{Multi-homing} refers to:
  \begin{enumerate}
    \item Hosting the platform in multiple cloud regions.
    \item Sellers or buyers participating in multiple platforms simultaneously.
    \item Using multiple payment gateways for redundancy.
    \item Maintaining multiple warehouses for the same SKU.
  \end{enumerate}

  \item Which is the \textbf{best} example of a marketplace \emph{governance} capability?
  \begin{enumerate}
    \item Re-ranking items using embeddings
    \item Seller onboarding verification and dispute resolution policies
    \item Demand forecasting for replenishment
    \item Offline model training with hyperparameter tuning
  \end{enumerate}
\end{enumerate}

\subsection*{Answer key}
\begin{enumerate}
  \item (b)
  \item (b)
  \item (b)
  \item (b)
  \item (b)
\end{enumerate}
}

\chapter{E-Commerce Data Foundations: Events, Experimentation, and Metrics (Week 3)}
\section{Scope (placeholder)}
\begin{itemize}
  \item Event schema design, identity resolution, attribution caveats, and data quality.
  \item A/B testing, bandits, and experimentation at scale (conceptual).
\end{itemize}
\section{Hands-on lab (placeholder)}
\begin{itemize}
  \item Build an event log + simple metrics layer for conversion and retention.
\end{itemize}

\section{Multiple-choice questions (MCQs)}
\begin{enumerate}
  \item In event-based analytics for e-commerce, which statement is most accurate?
  \begin{enumerate}
    \item Events should be stored without timestamps to save space.
    \item Events are most useful when they have a consistent schema, a timestamp, and stable identifiers.
    \item Only purchase events matter; click and view events are noise.
    \item The best schema changes frequently to match UI updates.
  \end{enumerate}

  \item Which is an example of a \emph{data quality} problem that can directly harm ML models?
  \begin{enumerate}
    \item A model uses a neural network instead of linear regression.
    \item Missing or duplicated events for key actions (e.g., add-to-cart) create biased labels/features.
    \item Using SQL instead of Python for ETL.
    \item Choosing a smaller batch size during training.
  \end{enumerate}

  \item Why are offline metrics often insufficient for evaluating an e-commerce ranking model?
  \begin{enumerate}
    \item Offline metrics are always higher than online metrics.
    \item Offline metrics cannot measure causal impact under real user behavior and feedback loops.
    \item Offline evaluation cannot be computed from logs.
    \item Offline metrics are not used in industry.
  \end{enumerate}

  \item In an A/B test, the primary purpose of randomization is to:
  \begin{enumerate}
    \item Increase model complexity.
    \item Ensure the treatment and control groups are comparable in expectation.
    \item Eliminate the need for monitoring.
    \item Guarantee the KPI improves.
  \end{enumerate}

  \item A common \emph{identity resolution} challenge in e-commerce is:
  \begin{enumerate}
    \item Mapping product IDs to SKU IDs.
    \item Linking the same person across devices/sessions while respecting privacy constraints.
    \item Computing NDCG@$k$ efficiently.
    \item Selecting a cloud region.
  \end{enumerate}
\end{enumerate}

\subsection*{Answer key}
\begin{enumerate}
  \item (b)
  \item (b)
  \item (b)
  \item (b)
  \item (b)
\end{enumerate}

\section{Exercises (short)}
\begin{enumerate}
  \item Propose an event taxonomy for a storefront: list at least 10 events (search, view, add-to-cart, checkout steps, purchase, return) and for each event specify 3--5 key fields.
  \item Define a North Star metric for an e-commerce product and break it into 3--5 leading indicators. Explain how you would compute each from event logs.
  \item Design a simple A/B test for a recommendation widget. Specify: unit of randomization, primary KPI, guardrail metrics, and an example of a bias/confounder to watch for.
\end{enumerate}

\section{Mini-case (Odoo-linked, vendor-neutral)}
\textbf{Scenario:} A hybrid B2C/B2B company uses Odoo (Sales, Inventory, Accounting, CRM). The storefront and mobile app generate clickstream events.

\textbf{Task:} Design a minimal data model that supports both analytics and ML:
\begin{itemize}
  \item Define how you will link clickstream identities to Odoo customers/partners (when possible) and how you handle guests.
  \item Propose which entities should have stable IDs across systems (customer, product, order, shipment) and which are channel-specific.
  \item Specify 5 ``data contracts'' (producer $\rightarrow$ consumer) that reduce breakage when the UI or ERP changes.
\end{itemize}

\chapter{Enterprise Architecture for E-Commerce (Week 4)}
\section{Scope (placeholder)}
\begin{itemize}
  \item Capability maps, value streams, domain boundaries, and target architecture.
  \item Reference views: business, application, data, and technology architectures.
\end{itemize}
\section{Artifacts (placeholder)}
\begin{itemize}
  \item Deliverable: e-commerce capability map + domain model.
\end{itemize}

\section{Multiple-choice questions (MCQs)}
\begin{enumerate}
  \item In enterprise architecture, a \emph{capability map} primarily describes:
  \begin{enumerate}
    \item A list of specific microservices and their endpoints.
    \item \emph{What} the business does (stable abilities), independent of \emph{how} it is implemented.
    \item The physical network topology of the data center.
    \item A Gantt chart of the project plan.
  \end{enumerate}

  \item A good domain boundary (e.g., for a ``Catalog'' domain) typically aims to:
  \begin{enumerate}
    \item Maximize shared database tables across all teams.
    \item Minimize coupling and define clear ownership of data and behaviors.
    \item Ensure all operations are synchronous.
    \item Avoid having APIs.
  \end{enumerate}

  \item Which artifact is most suitable for describing how business value is delivered end-to-end?
  \begin{enumerate}
    \item Value stream map
    \item Entity-relationship diagram (ERD)
    \item Source code repository layout
    \item TLS configuration file
  \end{enumerate}

  \item ``Target architecture'' is best described as:
  \begin{enumerate}
    \item The current-state system diagram.
    \item A future-state design that guides decisions and sequencing of change.
    \item A vendor product brochure.
    \item A test plan for unit tests.
  \end{enumerate}

  \item In an AI-driven e-commerce context, which is the best example of an \emph{architectural concern} (not a model choice)?
  \begin{enumerate}
    \item Whether to use XGBoost or logistic regression
    \item How to ensure training-serving consistency and monitor drift
    \item Whether to use $k$-means or DBSCAN
    \item Whether to use SGD or Adam
  \end{enumerate}
\end{enumerate}

\subsection*{Answer key}
\begin{enumerate}
  \item (b)
  \item (b)
  \item (a)
  \item (b)
  \item (b)
\end{enumerate}

\section{Exercises (short)}
\begin{enumerate}
  \item Draft a capability map for an AI-enabled e-commerce organization. Include at least: customer acquisition, discovery, pricing/promotions, order management, fulfillment/returns, customer support, data/analytics, and governance/security.
  \item Choose one value stream (e.g., ``order-to-cash'' or ``browse-to-buy''). Identify 5--8 steps and list the owning domain/system for each step (commerce vs ERP vs logistics vs payment).
  \item Propose domain boundaries for: catalog, pricing, orders, payments, and customer profiles. For each boundary, name the system-of-record for key data entities.
\end{enumerate}

\section{Mini-case (Odoo-linked, vendor-neutral)}
\textbf{Scenario:} Your company is hybrid B2C/B2B. Odoo is used for core ERP flows (Sales, Inventory, Accounting, CRM). A separate commerce layer handles web/mobile experiences and AI features.

\textbf{Task:} Produce a short ``target architecture'' note:
\begin{itemize}
  \item Draw a capability-to-system mapping (capabilities $\rightarrow$ commerce layer vs Odoo vs data platform).
  \item Define the \emph{system of record} for customer, product, price list, order, invoice, and shipment.
  \item Identify 3 integration risks (data consistency, latency, duplicate sources of truth) and propose mitigations.
\end{itemize}

\chapter{Enterprise Integration Patterns for E-Commerce (Week 5)}
\section{Scope (placeholder)}
\begin{itemize}
  \item API-first, integration styles (sync/async), event-driven architecture.
  \item iPaaS/ESB concepts, message brokers/streams, and reliability patterns.
\end{itemize}
\section{Hands-on lab (placeholder)}
\begin{itemize}
  \item Model an order lifecycle with events + idempotency + outbox pattern (conceptual + pseudo).
\end{itemize}


\section{Multiple-choice questions (MCQs)}
\begin{enumerate}
  \item Which statement best characterizes synchronous vs asynchronous integration?
  \begin{enumerate}
    \item Synchronous integration cannot fail; asynchronous always fails.
    \item Synchronous integration couples availability and latency; asynchronous trades immediacy for resilience and decoupling.
    \item Asynchronous integration is only for analytics; synchronous is only for ERP.
    \item They are equivalent as long as JSON is used.
  \end{enumerate}

  \item Why is idempotency important in order processing APIs?
  \begin{enumerate}
    \item It makes responses larger.
    \item It prevents duplicate effects when requests are retried.
    \item It eliminates the need for authentication.
    \item It guarantees exactly-once message delivery in all systems.
  \end{enumerate}

  \item The \emph{outbox pattern} is primarily used to:
  \begin{enumerate}
    \item Store images for the catalog.
    \item Reliably publish events when database writes succeed, avoiding dual-write inconsistencies.
    \item Encrypt API traffic.
    \item Replace a message broker.
  \end{enumerate}

  \item In event-driven systems, a \emph{consumer} should typically be designed to be:
  \begin{enumerate}
    \item State-less but non-retryable
    \item Idempotent and retryable
    \item Dependent on message ordering always being perfect
    \item Tightly coupled to producer database schemas
  \end{enumerate}

  \item Which is a common reason to choose events over direct API calls for some flows?
  \begin{enumerate}
    \item To avoid defining schemas/contracts
    \item To decouple systems and allow multiple downstream consumers without changing the producer
    \item To guarantee zero latency
    \item To avoid monitoring
  \end{enumerate}
\end{enumerate}

\subsection*{Answer key}
\begin{enumerate}
  \item (b)
  \item (b)
  \item (b)
  \item (b)
  \item (b)
\end{enumerate}

\section{Exercises (short)}
\begin{enumerate}
  \item For each flow, choose sync API or async events and justify: (i) ``place order'', (ii) ``update inventory availability'', (iii) ``invoice posted'', (iv) ``shipment delivered'', (v) ``customer profile updated''.
  \item Design an idempotency strategy for the ``create order'' API. Specify the idempotency key and how long it is retained.
  \item Write a short failure story: a dual-write bug between database and message broker. Explain how the outbox pattern prevents it.
\end{enumerate}

\section{Mini-case (Odoo-linked, vendor-neutral)}
\textbf{Scenario:} A headless storefront integrates with Odoo for orders, inventory, invoices, and CRM updates.

\textbf{Task:} Propose an integration design that supports scale and reliability:
\begin{itemize}
  \item Define the key domain events (at least 8) and which system publishes each.
  \item Identify where you need synchronous APIs (e.g., payment authorization) vs asynchronous events (e.g., shipment updates).
  \item Specify two data contracts and a versioning strategy to avoid breaking changes.
  \item List the minimum observability signals (logs/metrics/traces) required to operate the integration in production.
\end{itemize}

\section{Multiple-choice questions (MCQs)}
\begin{enumerate}
  \item Which statement best characterizes synchronous vs asynchronous integration?
  \begin{enumerate}
    \item Synchronous integration cannot fail; asynchronous always fails.
    \item Synchronous integration couples availability and latency; asynchronous trades immediacy for resilience and decoupling.
    \item Asynchronous integration is only for analytics; synchronous is only for ERP.
    \item They are equivalent as long as JSON is used.
  \end{enumerate}

  \item Why is idempotency important in order processing APIs?
  \begin{enumerate}
    \item It makes responses larger.
    \item It prevents duplicate effects when requests are retried.
    \item It eliminates the need for authentication.
    \item It guarantees exactly-once message delivery in all systems.
  \end{enumerate}

  \item The \emph{outbox pattern} is primarily used to:
  \begin{enumerate}
    \item Store images for the catalog.
    \item Reliably publish events when database writes succeed, avoiding dual-write inconsistencies.
    \item Encrypt API traffic.
    \item Replace a message broker.
  \end{enumerate}

  \item In event-driven systems, a \emph{consumer} should typically be designed to be:
  \begin{enumerate}
    \item State-less but non-retryable
    \item Idempotent and retryable
    \item Dependent on message ordering always being perfect
    \item Tightly coupled to producer database schemas
  \end{enumerate}

  \item Which is a common reason to choose events over direct API calls for some flows?
  \begin{enumerate}
    \item To avoid defining schemas/contracts
    \item To decouple systems and allow multiple downstream consumers without changing the producer
    \item To guarantee zero latency
    \item To avoid monitoring
  \end{enumerate}
\end{enumerate}

\subsection*{Answer key}
\begin{enumerate}
  \item (b)
  \item (b)
  \item (b)
  \item (b)
  \item (b)
\end{enumerate}

\section{Exercises (short)}
\begin{enumerate}
  \item For each flow, choose sync API or async events and justify: (i) ``place order'', (ii) ``update inventory availability'', (iii) ``invoice posted'', (iv) ``shipment delivered'', (v) ``customer profile updated''.
  \item Design an idempotency strategy for the ``create order'' API. Specify the idempotency key and how long it is retained.
  \item Write a short failure story: a dual-write bug between database and message broker. Explain how the outbox pattern prevents it.
\end{enumerate}

\section{Mini-case (Odoo-linked, vendor-neutral)}
\textbf{Scenario:} A headless storefront integrates with Odoo for orders, inventory, invoices, and CRM updates.

\textbf{Task:} Propose an integration design that supports scale and reliability:
\begin{itemize}
  \item Define the key domain events (at least 8) and which system publishes each.
  \item Identify where you need synchronous APIs (e.g., payment authorization) vs asynchronous events (e.g., shipment updates).
  \item Specify two data contracts and a versioning strategy to avoid breaking changes.
  \item List the minimum observability signals (logs/metrics/traces) required to operate the integration in production.
\end{itemize}

\chapter{ERP/CRM/SCM Integration in Commerce (Week 6)}
\section{Scope (placeholder)}
\begin{itemize}
  \item Odoo overview (high-level): key modules relevant to commerce (Sales, Inventory, Accounting, CRM).
  \item Order-to-cash, inventory, fulfillment, returns, and customer data flows.
  \item Master data management (MDM) and data ownership across domains.
\end{itemize}
\section{Artifacts (placeholder)}
\begin{itemize}
  \item Deliverable: integration context diagram + data contracts (high level), including an Odoo boundary.
\end{itemize}


\section{Multiple-choice questions (MCQs)}
\begin{enumerate}
  \item In an integrated commerce + ERP architecture, the \emph{system of record} is best defined as:
  \begin{enumerate}
    \item The system with the nicest UI
    \item The system that is most convenient for analytics
    \item The authoritative source of truth for a given entity (e.g., invoice) and its lifecycle
    \item The system that runs in the cloud
  \end{enumerate}

  \item Which flow is most strongly associated with \emph{order-to-cash}?
  \begin{enumerate}
    \item Creating a product image embedding
    \item Quote/order $\rightarrow$ delivery $\rightarrow$ invoice $\rightarrow$ payment reconciliation
    \item Building a recommendation model
    \item Defining a Kafka topic
  \end{enumerate}

  \item Master data management (MDM) is primarily concerned with:
  \begin{enumerate}
    \item Training deep learning models for search
    \item Defining and governing shared entities (customer, product, supplier) and preventing duplicates/inconsistencies
    \item Choosing the best database index
    \item Encrypting API requests
  \end{enumerate}

  \item A common integration risk when connecting commerce with ERP is:
  \begin{enumerate}
    \item Too many unit tests
    \item Duplicate sources of truth for prices, customers, or inventory
    \item Having an event schema
    \item Using idempotency keys
  \end{enumerate}

  \item In practice, why do integrations often require both APIs and events?
  \begin{enumerate}
    \item Because events are always cheaper than APIs
    \item Because some steps require immediate confirmation while others benefit from decoupling and retries
    \item Because ERP systems cannot expose APIs
    \item Because events guarantee zero duplicates without design effort
  \end{enumerate}
\end{enumerate}

\subsection*{Answer key}
\begin{enumerate}
  \item (c)
  \item (b)
  \item (b)
  \item (b)
  \item (b)
\end{enumerate}

\section{Exercises (short)}
\begin{enumerate}
  \item For each entity, choose a system of record (commerce vs Odoo) and justify: customer, product, price list, inventory on-hand, order, invoice, payment, return.
  \item Describe two failure modes of inventory sync and propose mitigations (e.g., eventual consistency, reservations, reconciliation jobs).
  \item Draft a ``data contract'' for an \texttt{OrderCreated} event: required fields, optional fields, and versioning rules.
\end{enumerate}

\section{Mini-case (Odoo-linked, vendor-neutral)}
\textbf{Scenario:} A hybrid company runs a headless storefront for B2C and a B2B portal for wholesale customers. Odoo is used for Sales, Inventory, Accounting, and CRM.

\textbf{Task:} Design the integration for three end-to-end processes:
\begin{itemize}
  \item \textbf{Order-to-cash:} from checkout to invoice posting and payment reconciliation.
  \item \textbf{Returns/refunds:} from return request to stock adjustment and credit note.
  \item \textbf{B2B pricing:} negotiated price lists and approvals without duplicating sources of truth.
\end{itemize}
For each process, state: key events, required synchronous calls, idempotency strategy, and the minimum monitoring you would implement.

\section{Multiple-choice questions (MCQs)}
\begin{enumerate}
  \item In an integrated commerce + ERP architecture, the \emph{system of record} is best defined as:
  \begin{enumerate}
    \item The system with the nicest UI
    \item The system that is most convenient for analytics
    \item The authoritative source of truth for a given entity (e.g., invoice) and its lifecycle
    \item The system that runs in the cloud
  \end{enumerate}

  \item Which flow is most strongly associated with \emph{order-to-cash}?
  \begin{enumerate}
    \item Creating a product image embedding
    \item Quote/order $\rightarrow$ delivery $\rightarrow$ invoice $\rightarrow$ payment reconciliation
    \item Building a recommendation model
    \item Defining a Kafka topic
  \end{enumerate}

  \item Master data management (MDM) is primarily concerned with:
  \begin{enumerate}
    \item Training deep learning models for search
    \item Defining and governing shared entities (customer, product, supplier) and preventing duplicates/inconsistencies
    \item Choosing the best database index
    \item Encrypting API requests
  \end{enumerate}

  \item A common integration risk when connecting commerce with ERP is:
  \begin{enumerate}
    \item Too many unit tests
    \item Duplicate sources of truth for prices, customers, or inventory
    \item Having an event schema
    \item Using idempotency keys
  \end{enumerate}

  \item In practice, why do integrations often require both APIs and events?
  \begin{enumerate}
    \item Because events are always cheaper than APIs
    \item Because some steps require immediate confirmation while others benefit from decoupling and retries
    \item Because ERP systems cannot expose APIs
    \item Because events guarantee zero duplicates without design effort
  \end{enumerate}
\end{enumerate}

\subsection*{Answer key}
\begin{enumerate}
  \item (c)
  \item (b)
  \item (b)
  \item (b)
  \item (b)
\end{enumerate}

\section{Exercises (short)}
\begin{enumerate}
  \item For each entity, choose a system of record (commerce vs Odoo) and justify: customer, product, price list, inventory on-hand, order, invoice, payment, return.
  \item Describe two failure modes of inventory sync and propose mitigations (e.g., eventual consistency, reservations, reconciliation jobs).
  \item Draft a ``data contract'' for an \texttt{OrderCreated} event: required fields, optional fields, and versioning rules.
\end{enumerate}

\section{Mini-case (Odoo-linked, vendor-neutral)}
\textbf{Scenario:} A hybrid company runs a headless storefront for B2C and a B2B portal for wholesale customers. Odoo is used for Sales, Inventory, Accounting, and CRM.

\textbf{Task:} Design the integration for three end-to-end processes:
\begin{itemize}
  \item \textbf{Order-to-cash:} from checkout to invoice posting and payment reconciliation.
  \item \textbf{Returns/refunds:} from return request to stock adjustment and credit note.
  \item \textbf{B2B pricing:} negotiated price lists and approvals without duplicating sources of truth.
\end{itemize}
For each process, state: key events, required synchronous calls, idempotency strategy, and the minimum monitoring you would implement.

\part{Machine Learning and Deep Learning for E-Commerce}

\chapter{Recommenders I: Classical Methods + Learning-to-Rank (Week 7)}
\section{Scope (placeholder)}
\begin{itemize}
  \item Collaborative filtering, matrix factorization, and implicit feedback.
  \item Learning-to-rank: pointwise/pairwise/listwise framing; offline vs online evaluation.
\end{itemize}
\section{Hands-on lab (placeholder)}
\begin{itemize}
  \item Baseline recommender with clear metrics (MAP@K/NDCG@K).
\end{itemize}

\chapter{Recommenders II: Deep Learning, Embeddings, and Retrieval (Week 8)}
\section{Scope (placeholder)}
\begin{itemize}
  \item Two-tower retrieval, embeddings, sequence models, and candidate generation vs re-ranking.
  \item Practical concerns: cold-start, diversity, bias, and latency.
\end{itemize}
\section{Hands-on lab (placeholder)}
\begin{itemize}
  \item Train embeddings; build a retrieval + re-ranking pipeline (high level).
\end{itemize}

\chapter{Pricing and Promotions with ML + Optimization (Week 9)}
\section{Scope (placeholder)}
\begin{itemize}
  \item Demand forecasting, elasticity, uplift, and constraints (inventory/competition).
  \item Optimization approaches and experiment design for pricing.
\end{itemize}
\section{Hands-on lab (placeholder)}
\begin{itemize}
  \item Forecast baseline + simple constrained optimization scenario.
\end{itemize}

\chapter{Fraud, Risk, and Trust \& Safety (Week 10)}
\section{Scope (placeholder)}
\begin{itemize}
  \item Fraud types: payment, account takeover, promotion abuse, refund abuse.
  \item Methods: supervised learning, anomaly detection, graph ML, and rule+ML hybrids.
\end{itemize}
\section{Hands-on lab (placeholder)}
\begin{itemize}
  \item Build a risk scoring baseline + thresholding and cost-sensitive evaluation.
\end{itemize}

\part{Applied AI Systems, LLMs, and Operations}

\chapter{Customer Service Automation with LLMs (Week 11)}
\section{Scope (placeholder)}
\begin{itemize}
  \item RAG, tool use, workflow orchestration, evaluation, and guardrails.
  \item When to use LLMs vs classical NLP vs rules.
\end{itemize}
\section{Hands-on lab (placeholder)}
\begin{itemize}
  \item Prototype: FAQ assistant with retrieval + citations (local documents only).
\end{itemize}

\chapter{MLOps + DataOps for E-Commerce (Week 12)}
\section{Scope (placeholder)}
\begin{itemize}
  \item Platform-agnostic MLOps/DataOps: CI/CD for ML, pipelines, and reproducibility.
  \item Feature stores (concept), training-serving skew, monitoring, drift, and SLOs.
  \item Experiment tracking, model governance, and evaluation at scale.
\end{itemize}
\section{Artifacts (placeholder)}
\begin{itemize}
  \item Deliverable: ML system design doc + monitoring plan.
\end{itemize}

\chapter{Security, Privacy, Compliance, and Responsible AI (Week 13)}
\section{Scope (placeholder)}
\begin{itemize}
  \item Payments/security basics, privacy-by-design, and data minimization.
  \item Threat modeling for e-commerce and AI features (LLM-specific risks included).
  \item Responsible AI: fairness, explainability, auditability, and human-in-the-loop controls.
\end{itemize}
\section{Discussion (placeholder)}
\begin{itemize}
  \item Risk assessment template for an AI feature.
\end{itemize}

\chapter{Capstone: Enterprise-Grade AI E-Commerce Architecture (Week 14)}
\section*{Capstone scenario (agreed)}
\begin{itemize}
  \item Scenario: end-to-end ``AI commerce + Odoo'' reference architecture for a single company.
  \item Goal: demonstrate EA alignment, ERP integration, and measurable AI value with operational readiness.
  \item Business model: hybrid (B2C + B2B).
  \item Commerce front-end: both (compare Odoo Website/eCommerce vs a separate headless storefront integrating to Odoo).
  \item AI stack: both (ML for recs/fraud/pricing + LLMs for support/ops).
\end{itemize}
\section{Scope (placeholder)}
\begin{itemize}
  \item End-to-end architecture: channels, services, data platform, ML/LLM layer, ERP integration.
  \item Evaluation: business metrics, technical metrics, and operational readiness.
\end{itemize}
\section{Capstone deliverables (placeholder)}
\begin{itemize}
  \item Architecture pack (diagrams), backlog, data contracts, and evaluation plan.
\end{itemize}


\appendix
\part{Appendices (Placeholders)}
\chapter{Datasets and Synthetic Data for Teaching (placeholder)}
\chapter{Template: System Design Document (placeholder)}
\chapter{Template: Data Contract (placeholder)}
\chapter{Template: Model Card and Risk Assessment (placeholder)}

\bibliographystyle{plainnat}
\bibliography{references}

\end{document}