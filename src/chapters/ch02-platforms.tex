\chapter{Platforms, Marketplaces, and Business Models (Week 2)}

\section{Learning objectives}
\begin{itemize}
  \item Distinguish e-commerce \emph{channels} (storefronts) from \emph{platforms} (ecosystems) and \emph{marketplaces} (multi-seller).
  \item Analyze platform economics: network effects, multi-homing, and pricing of participation.
  \item Translate business model choices into architectural requirements (identity, trust, data, integration, ERP).
\end{itemize}

\section{From storefront to platform}
\begin{itemize}
  \item \textbf{Storefront:} a single merchant selling to buyers (B2C/B2B) via a web/app channel.
  \item \textbf{Platform:} a system that enables interactions among multiple participant groups (e.g., buyers, sellers, advertisers, logistics providers).
  \item \textbf{Marketplace:} a platform where multiple sellers list items and transactions are mediated by platform policies and infrastructure.
\end{itemize}

\section{Value creation, value capture, and growth}
\subsection{Value creation and matching}
Platforms create value by reducing search and transaction costs, increasing variety, and improving trust; AI frequently strengthens these effects by improving matching and decision support.

\subsection{Value capture models}
\begin{itemize}
  \item Take rate (commission on completed transactions)
  \item Subscription (seller or buyer membership tiers)
  \item Listing fees and value-added services (e.g., promotions, analytics)
  \item Advertising (sponsored listings, retail media)
  \item Fulfillment fees (warehousing, last-mile delivery, returns handling)
\end{itemize}

\subsection{Network effects}
\begin{itemize}
  \item \textbf{Direct network effects:} more users attract more users (e.g., social commerce communities).
  \item \textbf{Indirect network effects:} more buyers attract more sellers and vice versa (classic marketplace dynamic).
  \item \textbf{Cold start:} bootstrapping supply and demand when network effects are weak.
  \item \textbf{Multi-homing:} sellers/buyers participate in multiple platforms; affects differentiation strategy.
\end{itemize}

\section{Governance and trust as system requirements}
Marketplace governance is not ``policy only'': it becomes a set of product and system requirements.
\begin{itemize}
  \item Seller onboarding, verification, and compliance workflows (KYC/KYB where relevant).
  \item Catalog integrity: listing policies, moderation, and counterfeit detection.
  \item Reviews/ratings, dispute resolution, refunds/returns policy enforcement.
  \item Trust \& safety operations: rule+ML detection, case management, and human-in-the-loop escalation.
\end{itemize}

\section{Business model patterns (and architectural implications)}
\subsection{B2C retail}
\begin{itemize}
  \item Key systems: PIM/catalog, promotions, payments, fulfillment, customer support.
  \item ERP/CRM integration: order-to-cash, inventory, accounting, returns.
\end{itemize}

\subsection{B2B commerce}
\begin{itemize}
  \item Quotes, negotiated pricing, approval workflows, invoicing, credit limits.
  \item Integration-heavy: procurement, supplier management, and EDI/API flows.
\end{itemize}

\subsection{Marketplace}
\begin{itemize}
  \item Split catalog ownership (platform vs sellers), seller tools, payout/settlement.
  \item Additional ``control plane'': seller policies, listing moderation, abuse prevention.
\end{itemize}

\subsection{Digital services (subscriptions)}
\begin{itemize}
  \item Recurring billing, entitlements, and churn/retention analytics.
  \item Usage-based pricing requires event design and metering.
\end{itemize}

\section{Where AI fits (system-level view)}
\begin{itemize}
  \item \textbf{Matching layer:} search, recommendations, ranking.
  \item \textbf{Trust layer:} fraud/abuse detection, content moderation.
  \item \textbf{Operations layer:} forecasting, replenishment, routing.
  \item \textbf{Experience layer:} personalization and conversational commerce.
\end{itemize}

\section{Hands-on (placeholder)}
\begin{itemize}
  \item Create a one-page \emph{platform blueprint}: participant groups, value exchange, pricing model, and required capabilities.
  \item Map capabilities to an EA view and identify which capabilities are ERP-owned vs commerce-owned.
\end{itemize}

\section{Case studies (placeholder)}
\begin{itemize}
  \item Modern marketplace patterns (generic; no vendor lock-in).
\end{itemize}

\section{Multiple-choice questions (MCQs)}
\begin{enumerate}
  \item Which statement best distinguishes a \emph{storefront} from a \emph{marketplace}?
  \begin{enumerate}
    \item A storefront is always B2B; a marketplace is always B2C.
    \item A storefront has a single merchant of record; a marketplace mediates transactions among multiple sellers and buyers.
    \item A storefront cannot use AI; a marketplace must use AI.
    \item A storefront requires ERP integration; a marketplace does not.
  \end{enumerate}

  \item \emph{Indirect network effects} in marketplaces most directly mean:
  \begin{enumerate}
    \item The platform has a single user group and growth is linear.
    \item The value to buyers increases as more sellers join, and the value to sellers increases as more buyers join.
    \item Users prefer multiple platforms (multi-homing) regardless of differentiation.
    \item The platform must subsidize logistics to succeed.
  \end{enumerate}

  \item In B2B commerce, which requirement is \textbf{most typical} compared to B2C?
  \begin{enumerate}
    \item Session-based recommendations
    \item Negotiated pricing and approval workflows
    \item Visual search
    \item Social reviews and influencer marketing
  \end{enumerate}

  \item \emph{Multi-homing} refers to:
  \begin{enumerate}
    \item Hosting the platform in multiple cloud regions.
    \item Sellers or buyers participating in multiple platforms simultaneously.
    \item Using multiple payment gateways for redundancy.
    \item Maintaining multiple warehouses for the same SKU.
  \end{enumerate}

  \item Which is the \textbf{best} example of a marketplace \emph{governance} capability?
  \begin{enumerate}
    \item Re-ranking items using embeddings
    \item Seller onboarding verification and dispute resolution policies
    \item Demand forecasting for replenishment
    \item Offline model training with hyperparameter tuning
  \end{enumerate}
\end{enumerate}

\subsection*{Answer key}
\begin{enumerate}
  \item (b)
  \item (b)
  \item (b)
  \item (b)
  \item (b)
\end{enumerate}
\chapter{Platforms, Marketplaces, and Business Models (Week 2)}

\section{Learning objectives}
\begin{itemize}
  \item Distinguish e-commerce \emph{channels} (storefronts) from \emph{platforms} (ecosystems) and \emph{marketplaces} (multi-seller).
  \item Analyze platform economics: network effects, multi-homing, and pricing of participation.
  \item Translate business model choices into architectural requirements (identity, trust, data, integration, ERP).
\end{itemize}


\section{Multiple-choice questions (MCQs)}
\begin{enumerate}
  \item Which statement best distinguishes a \emph{storefront} from a \emph{marketplace}?
  \begin{enumerate}
    \item A storefront is always B2B; a marketplace is always B2C.
    \item A storefront has a single merchant of record; a marketplace mediates transactions among multiple sellers and buyers.
    \item A storefront cannot use AI; a marketplace must use AI.
    \item A storefront requires ERP integration; a marketplace does not.
  \end{enumerate}

  \item \emph{Indirect network effects} in marketplaces most directly mean:
  \begin{enumerate}
    \item The platform has a single user group and growth is linear.
    \item The value to buyers increases as more sellers join, and the value to sellers increases as more buyers join.
    \item Users prefer multiple platforms (multi-homing) regardless of differentiation.
    \item The platform must subsidize logistics to succeed.
  \end{enumerate}

  \item In B2B commerce, which requirement is \textbf{most typical} compared to B2C?
  \begin{enumerate}
    \item Session-based recommendations
    \item Negotiated pricing and approval workflows
    \item Visual search
    \item Social reviews and influencer marketing
  \end{enumerate}

  \item \emph{Multi-homing} refers to:
  \begin{enumerate}
    \item Hosting the platform in multiple cloud regions.
    \item Sellers or buyers participating in multiple platforms simultaneously.
    \item Using multiple payment gateways for redundancy.
    \item Maintaining multiple warehouses for the same SKU.
  \end{enumerate}

  \item Which is the \textbf{best} example of a marketplace \emph{governance} capability?
  \begin{enumerate}
    \item Re-ranking items using embeddings
    \item Seller onboarding verification and dispute resolution policies
    \item Demand forecasting for replenishment
    \item Offline model training with hyperparameter tuning
  \end{enumerate}
\end{enumerate}

\subsection*{Answer key}
\begin{enumerate}
  \item (b)
  \item (b)
  \item (b)
  \item (b)
  \item (b)
\end{enumerate}

\section{Scope}
\begin{itemize}

  \item Multi-sided platforms, network effects, platform governance, and incentives.
  \item B2B e-commerce, procurement portals, and industry-specific marketplaces.\end{itemize}
\section{Core concepts for Master's students}
\subsection{From storefront to platform}
\begin{itemize}
  \item \textbf{Storefront:} a single merchant selling to buyers (B2C/B2B) via a web/app channel.
  \item \textbf{Platform:} a system that enables interactions among multiple participant groups (e.g., buyers, sellers, advertisers, logistics providers).
  \item \textbf{Marketplace:} a platform where multiple sellers list items and transactions are mediated by platform policies and infrastructure.
\end{itemize}

\subsection{Value creation and capture}
\begin{itemize}
  \item \textbf{Value creation:} reduce search/transaction costs, increase variety, improve trust, and improve matching (often AI-driven).
  \item \textbf{Value capture:} take rate, subscription, listing fees, ads, fulfillment fees, and data/analytics products.
\end{itemize}

\subsection{Network effects and growth loops}
\begin{itemize}
  \item \textbf{Direct network effects:} more users attract more users (e.g., social commerce communities).
  \item \textbf{Indirect network effects:} more buyers attract more sellers and vice versa (classic marketplace dynamic).
  \item \textbf{Cold start:} early-stage bootstrapping strategies (seed supply vs seed demand).
  \item \textbf{Multi-homing:} sellers/buyers participating in multiple platforms; implications for loyalty and differentiation.
\end{itemize}

\subsection{Governance and trust}
\begin{itemize}
  \item Identity and onboarding (KYC/KYB where relevant), seller verification, and catalog integrity.
  \item Reviews/ratings, dispute management, returns policy, and service-level enforcement.
  \item Trust \& safety as an \emph{operating system} for the marketplace: policies + detection + human-in-the-loop.
\end{itemize}

\section{Business model patterns (with architectural implications)}
\subsection{B2C retail}
\begin{itemize}
  \item Key systems: product information management (PIM), promotions, payments, fulfillment, customer service.
  \item ERP/CRM integration: order-to-cash, inventory, accounting, returns.
\end{itemize}

\subsection{B2B commerce}
\begin{itemize}
  \item Quotation, contracts, negotiated pricing, credit limits, invoicing, and approval workflows.
  \item Integration-heavy by nature: procurement, supplier management, and EDI/API flows.
\end{itemize}

\subsection{Marketplace}
\begin{itemize}
  \item Split catalog ownership (platform vs sellers), seller tools, and payout/settlement.
  \item Additional control plane: seller policies, listing moderation, and fraud/abuse prevention.
\end{itemize}

\subsection{Digital services (subscriptions)}
\begin{itemize}
  \item Recurring billing, entitlement management, and churn/retention analytics.
  \item Usage-based pricing requires event design and metering.
\end{itemize}

\section{AI is not ``a feature'': where it fits in the platform}
\begin{itemize}
  \item \textbf{Matching layer:} search, recommendations, and ranking across buyers and sellers.
  \item \textbf{Trust layer:} fraud detection, abuse detection, and content moderation.
  \item \textbf{Operations layer:} forecasting, replenishment, routing, and workforce planning.
  \item \textbf{Experience layer:} conversational commerce, assistants, and personalization.
\end{itemize}

\section{Hands-on (placeholder)}
\begin{itemize}
  \item Build a one-page \emph{platform blueprint}: participant groups, value exchange, pricing model, and required capabilities.
  \item Map the blueprint to a capability map and identify which capabilities are ERP-owned vs commerce-owned.
\end{itemize}

\section{Core concepts for Master's students}
\subsection{From storefront to platform}
\begin{itemize}
  \item \textbf{Storefront:} a single merchant selling to buyers (B2C/B2B) via a web/app channel.
  \item \textbf{Platform:} a system that enables interactions among multiple participant groups (e.g., buyers, sellers, advertisers, logistics providers).
  \item \textbf{Marketplace:} a platform where multiple sellers list items and transactions are mediated by platform policies and infrastructure.
\end{itemize}

\subsection{Value creation and capture}
\begin{itemize}
  \item \textbf{Value creation:} reduce search/transaction costs, increase variety, improve trust, and improve matching (often AI-driven).
  \item \textbf{Value capture:} take rate, subscription, listing fees, ads, fulfillment fees, and data/analytics products.
\end{itemize}

\subsection{Network effects and growth loops}
\begin{itemize}
  \item \textbf{Direct network effects:} more users attract more users (e.g., social commerce communities).
  \item \textbf{Indirect network effects:} more buyers attract more sellers and vice versa (classic marketplace dynamic).
  \item \textbf{Cold start:} early-stage bootstrapping strategies (seed supply vs seed demand).
  \item \textbf{Multi-homing:} sellers/buyers participating in multiple platforms; implications for loyalty and differentiation.
\end{itemize}

\subsection{Governance and trust}
\begin{itemize}
  \item Identity and onboarding (KYC/KYB where relevant), seller verification, and catalog integrity.
  \item Reviews/ratings, dispute management, returns policy, and service-level enforcement.
  \item Trust \& safety as an \emph{operating system} for the marketplace: policies + detection + human-in-the-loop.
\end{itemize}

\section{Business model patterns (with architectural implications)}
\subsection{B2C retail}
\begin{itemize}
  \item Key systems: product information management (PIM), promotions, payments, fulfillment, customer service.
  \item ERP/CRM integration: order-to-cash, inventory, accounting, returns.
\end{itemize}

\subsection{B2B commerce}
\begin{itemize}
  \item Quotation, contracts, negotiated pricing, credit limits, invoicing, and approval workflows.
  \item Integration-heavy by nature: procurement, supplier management, and EDI/API flows.
\end{itemize}

\subsection{Marketplace}
\begin{itemize}
  \item Split catalog ownership (platform vs sellers), seller tools, and payout/settlement.
  \item Additional control plane: seller policies, listing moderation, and fraud/abuse prevention.
\end{itemize}

\subsection{Digital services (subscriptions)}
\begin{itemize}
  \item Recurring billing, entitlement management, and churn/retention analytics.
  \item Usage-based pricing requires event design and metering.
\end{itemize}

\section{AI is not ``a feature'': where it fits in the platform}
\begin{itemize}
  \item \textbf{Matching layer:} search, recommendations, and ranking across buyers and sellers.
  \item \textbf{Trust layer:} fraud detection, abuse detection, and content moderation.
  \item \textbf{Operations layer:} forecasting, replenishment, routing, and workforce planning.
  \item \textbf{Experience layer:} conversational commerce, assistants, and personalization.
\end{itemize}

\section{Hands-on (placeholder)}
\begin{itemize}
  \item Build a one-page \emph{platform blueprint}: participant groups, value exchange, pricing model, and required capabilities.
  \item Map the blueprint to a capability map and identify which capabilities are ERP-owned vs commerce-owned.
\end{itemize}
\section{Case studies (placeholder)}
\begin{itemize}
}
  \item Modern marketplace patterns (generic, no vendor lock-in)\end{itemize}

\section{Multiple-choice questions (MCQs)}
\begin{enumerate}
  \item Which statement best distinguishes a \emph{storefront} from a \emph{marketplace}?
  \begin{enumerate}
    \item A storefront is always B2B; a marketplace is always B2C.
    \item A storefront has a single merchant of record; a marketplace mediates transactions among multiple sellers and buyers.
    \item A storefront cannot use AI; a marketplace must use AI.
    \item A storefront requires ERP integration; a marketplace does not.
  \end{enumerate}

  \item \emph{Indirect network effects} in marketplaces most directly mean:
  \begin{enumerate}
    \item The platform has a single user group and growth is linear.
    \item The value to buyers increases as more sellers join, and the value to sellers increases as more buyers join.
    \item Users prefer multiple platforms (multi-homing) regardless of differentiation.
    \item The platform must subsidize logistics to succeed.
  \end{enumerate}

  \item In B2B commerce, which requirement is \textbf{most typical} compared to B2C?
  \begin{enumerate}
    \item Session-based recommendations
    \item Negotiated pricing and approval workflows
    \item Visual search
    \item Social reviews and influencer marketing
  \end{enumerate}

  \item \emph{Multi-homing} refers to:
  \begin{enumerate}
    \item Hosting the platform in multiple cloud regions.
    \item Sellers or buyers participating in multiple platforms simultaneously.
    \item Using multiple payment gateways for redundancy.
    \item Maintaining multiple warehouses for the same SKU.
  \end{enumerate}

  \item Which is the \textbf{best} example of a marketplace \emph{governance} capability?
  \begin{enumerate}
    \item Re-ranking items using embeddings
    \item Seller onboarding verification and dispute resolution policies
    \item Demand forecasting for replenishment
    \item Offline model training with hyperparameter tuning
  \end{enumerate}
\end{enumerate}

\subsection*{Answer key}
\begin{enumerate}
  \item (b)
  \item (b)
  \item (b)
  \item (b)
  \item (b)
\end{enumerate}
}
