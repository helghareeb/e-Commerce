\chapter{E-Commerce Data Foundations: Events, Experimentation, and Metrics (Week 3)}
\section{Scope (placeholder)}
\begin{itemize}
  \item Event schema design, identity resolution, attribution caveats, and data quality.
  \item A/B testing, bandits, and experimentation at scale (conceptual).
\end{itemize}
\section{Hands-on lab (placeholder)}
\begin{itemize}
  \item Build an event log + simple metrics layer for conversion and retention.
\end{itemize}

\section{Multiple-choice questions (MCQs)}
\begin{enumerate}
  \item In event-based analytics for e-commerce, which statement is most accurate?
  \begin{enumerate}
    \item Events should be stored without timestamps to save space.
    \item Events are most useful when they have a consistent schema, a timestamp, and stable identifiers.
    \item Only purchase events matter; click and view events are noise.
    \item The best schema changes frequently to match UI updates.
  \end{enumerate}

  \item Which is an example of a \emph{data quality} problem that can directly harm ML models?
  \begin{enumerate}
    \item A model uses a neural network instead of linear regression.
    \item Missing or duplicated events for key actions (e.g., add-to-cart) create biased labels/features.
    \item Using SQL instead of Python for ETL.
    \item Choosing a smaller batch size during training.
  \end{enumerate}

  \item Why are offline metrics often insufficient for evaluating an e-commerce ranking model?
  \begin{enumerate}
    \item Offline metrics are always higher than online metrics.
    \item Offline metrics cannot measure causal impact under real user behavior and feedback loops.
    \item Offline evaluation cannot be computed from logs.
    \item Offline metrics are not used in industry.
  \end{enumerate}

  \item In an A/B test, the primary purpose of randomization is to:
  \begin{enumerate}
    \item Increase model complexity.
    \item Ensure the treatment and control groups are comparable in expectation.
    \item Eliminate the need for monitoring.
    \item Guarantee the KPI improves.
  \end{enumerate}

  \item A common \emph{identity resolution} challenge in e-commerce is:
  \begin{enumerate}
    \item Mapping product IDs to SKU IDs.
    \item Linking the same person across devices/sessions while respecting privacy constraints.
    \item Computing NDCG@$k$ efficiently.
    \item Selecting a cloud region.
  \end{enumerate}
\end{enumerate}

\subsection*{Answer key}
\begin{enumerate}
  \item (b)
  \item (b)
  \item (b)
  \item (b)
  \item (b)
\end{enumerate}

\section{Exercises (short)}
\begin{enumerate}
  \item Propose an event taxonomy for a storefront: list at least 10 events (search, view, add-to-cart, checkout steps, purchase, return) and for each event specify 3--5 key fields.
  \item Define a North Star metric for an e-commerce product and break it into 3--5 leading indicators. Explain how you would compute each from event logs.
  \item Design a simple A/B test for a recommendation widget. Specify: unit of randomization, primary KPI, guardrail metrics, and an example of a bias/confounder to watch for.
\end{enumerate}

\section{Mini-case (Odoo-linked, vendor-neutral)}
\textbf{Scenario:} A hybrid B2C/B2B company uses Odoo (Sales, Inventory, Accounting, CRM). The storefront and mobile app generate clickstream events.

\textbf{Task:} Design a minimal data model that supports both analytics and ML:
\begin{itemize}
  \item Define how you will link clickstream identities to Odoo customers/partners (when possible) and how you handle guests.
  \item Propose which entities should have stable IDs across systems (customer, product, order, shipment) and which are channel-specific.
  \item Specify 5 ``data contracts'' (producer $\rightarrow$ consumer) that reduce breakage when the UI or ERP changes.
\end{itemize}
